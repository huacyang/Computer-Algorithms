\documentclass{article}
\usepackage{amssymb}
\usepackage{amsmath}
\usepackage{amsfonts}
\usepackage{latexsym}
\usepackage{times}
%\usepackage{psfrag,psfig,epsfig,epsf}
\usepackage{graphics}
\usepackage{multirow}
\usepackage{fullpage}
\usepackage{verbatim}
\usepackage{fancyheadings}
\usepackage[T1]{fontenc}
\usepackage{arev}
\usepackage{subfigure}
\usepackage{url}
\usepackage[noline,noend,ruled,linesnumbered]{algorithm2e}
\usepackage{algpseudocode}
\linespread{1.02} 

\pagestyle{empty}

\addtolength{\topmargin}{-20pt}
\addtolength{\oddsidemargin}{-5pt}
\addtolength{\textwidth}{20pt}
\addtolength{\textheight}{50pt}

\newenvironment{myitem}{\begin{list}{$\bullet$}
{\setlength{\itemsep}{-0pt}
\setlength{\topsep}{0pt}
\setlength{\labelwidth}{0pt}
%\setlength{\labelsep}{0pt}
\setlength{\leftmargin}{10pt}
\setlength{\parsep}{-0pt}
\setlength{\itemsep}{0pt}
\setlength{\partopsep}{0pt}}}%
{\end{list}}

\begin{document}

\sloppy

\noindent \underline{CS 344: DESIGN AND ANALYSIS OF COMPUTER
  ALGORITHMS \hspace{1.6in} SPRING 2014}

\vspace{0.1in}

\begin{center}
{\bf {\large Homework 2}}\\
Divide-and-Conquer Algorithms, Sorting Algorithms, Greedy Algorithms\\
\end{center}

\vspace{0.1in}

\noindent Deadline: March 14, 11:59pm.\\ 
Available points: 110. Perfect score: 100.\\

\vspace{0.1in}

\noindent{\bf {\large Hua Yang}}\\
{\bf {\large Alexio Mota}}\\
{\bf {\large Erik Kamp}}\\

\begin{center}
{\bf Homework Instructions:}
\end{center}

\vspace{0.1in}

{\bf }

\begin{center}
{\bf Part A (20 points)}
\end{center}

\noindent {\bf Problem 1:} The more general version of the Master
Theorem is the following. Given a recurrence of the form: 
$$T(n) = a T(\frac{n}{b}) + f(n)$$
where $a \geq 1$ and $b > 1$ are constants and $f(n)$ is an
asymptotically positive function, there are 3 cases: 
\begin{enumerate}
\item If $f(n) = O(n^{log_ba - \epsilon}$) for some constant $\epsilon
  > 0$, then $T(n) = \Theta(n^{log_ba})$.
\item If $f(n) = \Theta(n^{log_ba} log^kn)$ with $k \geq 0$, then
  $T(n) = \Theta(n^{log_ba} log^{k+1}n)$. In most cases, $k = 0$.
\item If $f(n) = \Omega(n^{log_ba+\epsilon})$ with $\epsilon > 0$, and
  $f(n)$ satisfies the regularity condition, then $T(n) = \Theta( f(n)
  )$. The regularity condition specifies that $a f(\frac{n}{b}) \leq c
  f(n)$ for some constant $c < 1$ and all sufficiently large $n$.
\end{enumerate}

\noindent Give asymptotic bounds for the following recurrences. Assume
$T(n)$ is constant for $n = 1$. Make your bounds as tight as possible,
and justify your answers.\\

\noindent A. $T(n) = 2T(\frac{n}{4}) + n^{0.51}$\\

\textbf{ Answer }
\begin{itemize}
\item $af(n/b) < cf(n) → a = 2 b = 4 → 2f(n/4) < cf(n) where c < 1$
\item $T(n) = \Omega(f(n)) → T(n) = 2T(n/4) + n0.51 = n0.51$
\item $\Omega(nlogb(a+e)) → \Omega(nlog4(2))+e) = f(n)$
\item $\Omega(nlogb(a+e)) → \Omega(n0.5+e) = n0.51$  which proves that $f(n) = \Omega(nlogb(a)+e)$ where $e < .01$
\item Therefore we can apply case 3 in this situation and get that $T(n) = \Theta(n0.51)$
\end{itemize}

\noindent B. $T(n) = 16 T(\frac{n}{4}) + n!$\\

\textbf{ Answer }
\begin{itemize}
\item Apply Case 3
\item $T(n) = \Omega(f(n)) → T(n) = 16T(n/4) + n! = n!$
\item $f(n) = \Omega(nlogb(a) + e)  →  n! = Ω(nlog4(16) + 3) → n! = \Omega(n2+e)$
\item By condition: $16f(n/4)  <= cf(n) where c < 1$.  
\item We apply case 3 and get that $T(n) = \Theta(n!)$
\end{itemize}

\noindent C. $T(n) = \sqrt{2} T(\frac{n}{2}) + logn$\\

\textbf{ Answer }
\begin{itemize}
\item Case 1 $\Theta(\sqrt{n})$
\item $f(n) = O(n^{logb(a)-e})$ where $a = \sqrt{2}$  and $b = 2$ and $f(n) = log(n)$
\item $O(n^{log2(\sqrt{2})+e}) = O(n^{12 - e}) =F(log(n))$
\item $O(n^{1/2 - e}) = f(log(n))$ we can say from the book that any polynomial dominates any logarithm. (Common sense rules page 16 rule 4)
\item We apply case 1 and get that $T(n) = \Theta(n)$
\end{itemize}

\noindent D. $T(n) = T(n-1) + lgn$\\

\textbf{ Answer }
\begin{itemize}
\item Apply case 
\item $a = 1$ ,$b = ?$, $log_{b}(a) = 0$ since $a = 1$, (aka: $b^{0} = 1$)
\item $f(n) = log(n) = \Theta(n^{logb(a)}log^{k}(n))$
\item Where k is 1
\item $\Theta(n^{0}log^{1}(n)) = \Theta(log(n))$
\item Therefore we can say that $f(n) = (n^{logb(a)}log^{k}(n))$ because $log(n) = \Theta(log(n))$
\item Therefore we apply case 2 to get $T(n) = \Theta(n^{logb(a)}log^{k+1}(n)) = \Theta(log^{k+1}(n)) = \Theta(log^{2}(n))$
\end{itemize}

\noindent E. $T(n) = 5T(n/5) + \frac{n}{lgn}$\\

\textbf{ Answer }
\begin{itemize}
\item $a = 5$, $b = 5$ and $f(n) = n/(log(n))$
\item $n^{log5(5)} → n^{1} → n^{1(+-)e}$ 
\item $n/log(n) = nlog^{k}(n)$ where most time $k = 0$ then → $n/log(n) = \Theta(n)$
\item $n/log(n) = O(n^{1-e})$
\item Therefore we can apply case 1 since $n/log(n)$ is upper bounded by $n^{1-e}$. From this case we can get that $T(n) = \Theta(n^{logb(a)}) = \Theta(n^{1})$
\end{itemize}


\begin{center}
{\bf Part B (25 points)}
\end{center}

\noindent {\bf Problem 2:} You are in the HR department of a
technology firm, and here is a job for you.  There are $n$ different
projects, and $n$ different programmers.

Every project has its unique payoff when completed and level of
difficulty (which are uniform, regardless which programmer will work
on the project).  Every programmer has a unique skill set as well as
expectations for compensation (which are uniform, regardless the
project the programmer will work on). You cannot directly collect
information that allows you to compare the payoff or difficulty level
of two projects, or the capability or expectations for compensation of
two programmers.

Instead, you can arrange a meeting between each project manager and
programmer.  In each meeting, the project manager will give the
programmer an interview to see whether the programmer can do the
project; the programmer can ask the project manager about the
compensation to see whether it meets her expectations.  After the
meeting, you can get a result based on the feedback of the project
manager and the programmer.  The result can be:

\begin{enumerate}
  \item The programmer can't do the project.
  \item The programmer can do the project, but the compensation of the
    project doesn't meet her expectation.
  \item The programmer can do the project, and the compensation for
    the project matches her expectations. At this time, we say the
    project and the programmer \textit{match} with each other.
\end{enumerate}

Assume that the projects and programmers match one to one. Your goal
is to match each programmer to a project.

{\bf A.} Show that any algorithm for this problem must need $\Omega( n
\log{n} )$ meetings in the worst case.\\

\textbf{ Answer }
\begin{itemize}
\item with x number of project managers and x number of programmers, there are x! permutations
\item use binary search as the algorithm, there are at most $2n$ leaves where n is the height of the tree in worst case
\item $2^{n} >= x!$
\item $n >= log_{2}x!$	, by taking the $log_{2}$ of both sides
\item since $log_{2}x! = (x log_{2}x)$
\item therefore, $n = \Omega(n log_{2}n)$
\end{itemize}

{\bf B.}  Design a randomized algorithm for this problem that runs in
expected time $O ( n \log{n} )$.\\

\textbf{ Answer }
\begin{itemize}
\item apply Master Theorem
\item pick a project manager and partition the developers with respect to this project manager, in order to find a match
\item pick a developer and partition the project managers with respect to this developer, in order to find a match
\item for each partition, split the selected group of developers (or project managers depending on the case) into two, stop once a match is found
\item $T(n) = T(x) + T(n - x) + O(n)$, by spliting into 2 sub-problems
\item average run-time is $O(n log n)$
\end{itemize}

\begin{center}
{\bf Part C (40 points)}
\end{center}


\noindent {\bf Problem 3:} A nation-wide programming contest is held
at $k$ universities in North America. The $i^{th}$ university has
$m_{i}$ participants. The total number of participants is $n$, i.e.,
$n = \sum_{i=1}^{k}m_{i}$. In the contest, participants have to write
programs to solve 6 problems. Each problem contains 10 test cases,
each test case is worth 10 points. Participants aim to maximize their
collected points.  

After the contest, each university sorts the scores of participants
belonging to it and submits the grades to the organizer.  Then the
organizer has to collect the sorted scores of participants and provide
a final sorted list for all participants.

\begin{enumerate}

\item For each university, how do they sort the scores of participants
  belonging to it? Please briefly describe a comparative sorting
  algorithm that is appropriate for this purpose and a non-comparative
  sorting algorithm that works in this setup.\\
  
\textbf{ Answer }
\begin{itemize}
\item Comparative : Merge sort should be used for the comparative method of sorting because it will only take O(nlog(n)) time to sort a list of n scores. Whereas quicksort will have a worst case runtime of O(n2). Additionally Merge sort works best with sequentially accessed elements like a stack of scores and will preserve the order of the scores.

\item Non-Comparative : Radix sort should be used for the non-comparative method. Radix sort works for this situation because scores can be separated by their integer or digit number. In this sort the scores will be collected and then separated by their value into k groups or buckets. The groups or buckets will be based on the least or most significant digit of the score. These groups are based on the value of the score so if two participants have the same score they will be in the same group. This non-comparative sort will be faster than the comparative sort only taking O(nk) time where k is the number of groups or buckets and n is the number of digits.
\end{itemize}

\item How does the organizer sort the scores of participants given $k$
  files, where each file includes the sorted scores of participants
  from a specific university?  Please describe an algorithm with a
  $O(n\log k)$ running time and justify its time complexity.\\
  
\textbf{ Answer }
\begin{itemize}
\item The algorithm to sort the scores of participants given k files would involve a min-heap. You start off with your input lists which can be viewed as a 2D array of scores.

\item We initialize the heap with stores index j for a list and the index i where i,j correspond to a score in list j. The heap has functions insert(i,j) and deleteMin() which returns a i,j pair. We start off by populating the heap with the pair 0,j so the heap will have k entries where k is the number of lists. This runs $O(k)$ times, inserting into the heap would take $O(logk)$, so total runtime is $O(k*logk)$.

\item Now we will proceed to delete the min score from the heap and insert into a result list, result. The min score will be inserted at index c where c starts at 0 and is incremented after each insertion. This will be a loop that runs $O(n)$ times where each iteration a min pair i,j is deleted from the heap and inserted into the result list, result. Deletion from the heap would take $O(logk)$. Insertion and deletion in each iteration would take $O(logk+logk)$

\item Once we delete a min pair from the heap, we will insert i++,j from list j corresponding to the delete index i score into the heap. The reason we do this is because since we have deleted the min score represented by i from the heap, we assume that the next smallest value can be the score in j at index i++ since it is sorted. Pair i++,j will only be inserted if it is not the end of the list.

\item The run time would be $O(n(logk+logk) ) = O(nlogk)$.
\end{itemize}

\item Suppose the organizer want to figure out the participants of
  ranking $r$. Given $k$ sorted files, how does the organizer find the
  $r^{th}$ largest score without sorting the scores of participants?
  (e.g. if r = 5 you need to compute the $5^{th}$ largest score, not 
  the top 5 scores). Please describe an algorithm in $O(k(\sum_{i=1}^{k}\log m_{i}))$ 
  time and justify its time complexity. 

Could you do this in $O(\log k (\sum_{i=1}^{k}\log m_{i} + k))$ time? 

  [Hint: If $r$ is smaller than $\dfrac{n}{2}$, the elements that have at least $\dfrac{n}{2}$ 
  elements smaller than them should not be possible answer. The problem is how to identify these
  elements.]\\
  
\textbf{ Answer }
\begin{itemize}
\item Assuming that all the lists are sorted in the same order in this case least to greatest the organizer can start off by looking at the last element or score in each sorted list. The organizer must look at the last score in each list erasing a previously greater score when they come to an even greater score (like if the greatest was 50 and comes across a 70). Once the organizer reaches the end of the list they note down that rounds largest and move the end of the list where they found the greatest down / forward one so that they do not repeat that number. The organizer repeats this process until they've reached their desired rth largest score.

\item Proceedure : 

for r != 0 ; r- -\\
\{

\qquad //find greatest from list of sorted

\qquad for i = 0; i < numberOfLists; i++

\qquad \{

\qquad \qquad //Get the element in the last pos of the list

\qquad \qquad if ((listsOfLists[i]).get(listOfLists[i].lastListPointer) > currentMax)

\qquad \qquad \{

\qquad \qquad \qquad currentMax =((listsOfLists[i]).get(listOfLists[i].lastListPointer)

\qquad \qquad \}

\qquad \}

\qquad increaseEndOfListContaining(currentMax)

\}\\
return currentMax\\

\item This algorithm will use k binary searches looking for the rth largest from k sorted lists. Because binary search takes $O(log(n))$ time the total time should take k time to look through the k lists last element and log(n) time to search through the k elements each loop finding the largest. Therefore the total runtime will wind up taking $O(k * SumOf( log(m_{i}), 0 <= i <= k)$ time where there are k lists and n elements in total.
\end{itemize}

Can you do this in $O(k+\log k \sum_{i=1}^{k}\log m_{i}+k\log k) = O(\log k (\sum_{i=1}^{k}\log m_{i}+k))$ time?

\textbf{ Answer }
\begin{itemize}
\item I do not know
\end{itemize}

\end{enumerate}


\noindent {\bf Problem 4:} You have a collection of $n$ New York Times
crossword puzzles from 01/01/1943 until 12/31/2012 stored in a
database. The only operations that you can perform to the database are
the following:
\begin{itemize}
\item crossword\_puzzle $x$ $\leftarrow$ getPuzzle( int index ); where the
  index is between $1$ and $n$; the puzzles are \emph{not} sorted in the
  database in terms of the date they appeared.
\item getDay( crossword\_puzzle x ); which returns a number between 1
  to 31.
\item getMonth( crossword\_puzzle x ); which returns a number between
  1 to 12.
\item getYear( crossword\_puzzle x ); which returns a number between
  1943 to 2012.
\end{itemize}
All of the above queries can be performed in constant time. You have
found out that the number of puzzles is less than the number of days
in the above period (from 01/01/1943 until 12/31/2012) by one, i.e.,
one crossword puzzle was not included in the database. We need to
identify the date of the missing crossword puzzle.

Design a linear-time algorithm that minimizes the amount of space that
it is using to find the missing date. Ignore the effect of leap years.\\

\textbf{ Answer }
\begin{itemize}
\item Let n be the number of dates between 01/01/1943 and 12/31/2012. We use three variables ( TotalDaySum, TotalMonthSum, TotalYearSum) to represent the sum of the values from 01/01/1943 and 12/31/2012. For example, TotalMonthSum represents the sum of months 01-12 for each year. The sum for one year would be 1+2+3+4+...+12 =  78. To the TotalMonthSum would be 78*(100-43+12)=78(69)=5382. Now if 1 date is missing, we can get the MonthSum for the available dates and perform TotalMonthSum-MonthSum to get the month of the missing date. For example, if the missing data is 09/12/2001 the MonthSum would be 5373. By performing TotalMonthSum-MonthSum you get 5382-5373=9 which is the month of the missing date. Repeat this process for the daysum and the year sum to get the full date. 

\item Pseudocode:

TotalDaySum, TotalMonthSum, TotalYearSum\\

// Go through dates 01/01/1943 and 12/31/2012

getTotal(dates): 

\qquad for(i -> dates.length)  //takes O(n)

\qquad \qquad day, month, year at dates[i]

\qquad \qquad TotalDaySum <- TotalDaySum + day

\qquad \qquad TotalMonthSum <- TotalMonthSum + month

\qquad \qquad TotalYearSum <- TotalYearSum + year\\
	
DaySum, MonthSum, YearSum

for( i -> n): // takes O(n)

\qquad crosswordPuzzle x <- getPuzzle(i)

\qquad day <- getDay(x)

\qquad month <- getMonth(x)

\qquad year <- getYear(x)

\qquad DaySum = DaySum + day\\

\qquad MonthSum <- MonthSum + month

\qquad YearSum <- YearSum + year

endfor\\

missingDay <- TotalDaySum - DaySum

missingMonth <- TotalMonthSum - MonthSum

missingYear <- TotalYearSum - YearSum\\

MissingDate = missingDay+”/” + missingMonth + “/” + missingYear

\end{itemize}


\begin{center}
{\bf Part D (25 points)}
\end{center}

\noindent {\bf Problem 5:} You are running a promotional event for a
company during which the plan is to distribute $n$ gifts to the
participants. Consider that each gift $i$ is worth an integer number
of dollars $a_i$. There are $m$ people participating in the event,
where $m < n$. The $j$-th person is satisfied if he receives gifts
that are worth at least $s_j$ dollars each. The task is to satisfy as
many people as possible given that you have a knowledge of the gift
amounts $a_i$ and the satisfaction requirements of each person
$s_j$. Give an approximation algorithm for assigning rewards to people
with a running time of $O(m\log m+n)$. What is the approximation ratio
of your algorithm and why?\\

\textbf{ Answer }
\begin{itemize}
\item Since we have m people attending the promotional event and n gifts we can do the following:

\begin{itemize}
\item Sort the m people by their satisfaction requirements, which if using merge sort will take $O(mlog(m))$ time. 

\item Secondly we radix sort all the gifts by their gift values. Radix sort will take $O(nk)$ time where k is the number of groups and n is the number of presents. However we can state that n > k through the assumption that the gifts have similar values and are not too sparse. Therefore we can drop this runtime down to $O(n)$.

\item Our approximation value comes in when selecting the correct gift for the correct participant. When selecting the gift we will use the greedy algorithm selecting a gift that is closest to the participants satisfaction value. Once the gift is selected for the participant subtract the gifts value and repeat on the same participant until the satisfaction value for the participant is less than or equal to 0.
\end{itemize}

\item Total runtime : sorting participants : $O(mlog(m))$ + radix Gifts $O(n) = O(mlog(m) + n)$
\end{itemize}


\end{document}

